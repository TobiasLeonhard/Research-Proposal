\chapter{Introduction to Continuous Permafrost}
Permafrost is overlain by a ground layer that thaws and refreezes seasonally, called the active layer. 
Surface and near-surface water movement and storage are mostly controlled by this layer~\citep{walvoordHydrologicImpactsThawing2016}. 
The active layer also plays an important role in isotope cycling~\citep{tetzlaffTracerbasedAssessmentFlow2015}.

In permafrost, water flow through the ground can be divided into three flow zones. 
Suprapermafrost aquifers, subpermafrost aquifers, and intrapermafrost groundwater~\citep{walvoordHydrologicImpactsThawing2016,wooPermafrostHydrology2012}. 
Subpermafrost aquifers are naturally occurring due to the geothermal gradient, suprapermafrost aquifers are all aquifers that develop between permafrost and the surface, examples for this are the active layer and taliks (permanently unfrozen zones in permafrost), intrapermafrost groundwater can for example develop if permafrost aggregates over an talik, for example after a lake drainage~\citep{walvoordHydrologicImpactsThawing2016,wooPermafrostHydrology2012}.

In general, taliks can connect the different flow zones, allowing for mixing of different groundwaters. However, this process gets more unlikely the thicker (and the more continuous) permafrost becomes~\citep{wooPermafrostHydrology2012}. Therefore, permafrost can become an impermeable barrier between suprapermafrost and subpermafrost aquifers. However, water still can infiltrate permafrost; this for example happens during the formation and growth of \glspl{iwp}, which will be discussed in further detail in the following section.

The combination of the lacking exchange between suprapermafrost and subpermafrost aquifers in deep continuous permafrost, the time needed to develop the active layer, the general main water mobilization events (snowmelt and late summer precipitation) in the Arctic, and the hydraulic conductivity of the ground (which is also affected by the ground temperature) makes groundwater flow behavior in continuous permafrost regions unique. 
This combination can be used to explain the slow release of radionuclides, i.e.\ the high residuum of legacy tritium in permafrost measured by \citet{bondPermafrostThawImplications2018}. However, it does not explain the accumulation of radionuclides in the first place. For example, it is not known how uniform the radionuclide accumulation in permafrost was before the main thermonuclear weapons testing period in the 1950s and 1960s. 
Furthermore, \citet{quintonSubsurfaceDrainageHummockcovered2000} discussed that the uppermost organic layer can have a high hydraulic conductivity, even while frozen, allowing a higher replenishing rate for the upper surface. 
Such higher hydraulic conductivity in frozen soils can only occur if the soil was undersaturated enough before freezing, which can occur due to dry soil during the refreezing process or as water gets sucked down and upwards by the two freezing fronts while refreezing~\citep{mackayOriginHummocksWestern1980}.

The accumulation of radionuclides in permafrost could be explained by the processes involved in the formation of \glspl{iwp}, mineral earth hummocks, and lake bottom sediments; the latter are often taliks. As those processes provide the opportunity for surface water and pollutants to penetrate deeper into the ground, be it through sediment accumulation (lakes) or ice cracking (for \glspl{iwp}).\@

\section[Ice-wedge polygons (IWPs)]{Ice-wedge polygons (\glspl{iwp})}
\glspl{iwp} are distinct permafrost landforms characterized by a polygonal network of surface troughs or ridges. 
The center of \glspl{iwp} is located between the ice wedge structure creating the \glspl{iwp}. 
For the formation of ice wedges ice veins, sufficiently low temperatures, and sufficiently large timescales (from centuries to millennia) are needed. 
Ice wedges are wedge-like structures mostly consisting of ice that can reach dimensions of up to 5 meters in width and 20 meters in depth~\citep{ulrichQuantifyingWedgeIceVolumes2014}.

On sufficient cold winter days, ice veins in permafrost can crack. 
Those cracks are then filled by water primarily during the melting seasons, that seeps into the permafrost and refreezes in the crack~\citep{fortierLateHoloceneSyngenetic2004,liljedahlPanArcticIcewedgeDegradation2016}. 
Through this process, the size of the ice veins increases, eventually becoming an ice wedge. The amount and timing of the cracking process depends on the thermal contraction coefficient of the ice-soil mixture, the temperature, and temperature duration~\citep{fortierLateHoloceneSyngenetic2004}. 
This growing ice wedge applies vertical and horizontal stress on the surrounding soil, leading to vertical and horizontal cryoturbations~\citep{fortierLateHoloceneSyngenetic2004,liljedahlPanArcticIcewedgeDegradation2016}. These cryoturbations lead to a polygonal relief with \glspl{lcp}. 
\glspl{lcp} are therefore indicating growing \glspl{iwp} and can be recognized by a polygonal arranged pond-like structure with small throughs in the pond rims; the ponds are thereby located in the horizontal center of the \glspl{iwp}.\@
If the ice wedges degrade, for example through an deepening active layer, subsidence occurs, eventually creating \glspl{hcp}. 
\glspl{hcp} have therefore the inverse topography of \glspl{lcp} and indicate the degeneration of the ice wedges. They can be recognized by relatively dry mounds that are segregated by relatively wet throughs, where the mounds are located at the center of the \glspl{iwp}.\@

The ongoing geocryological debate regarding the precise mechanism of \gls{lcp} formation is summarized in the recent review by \citet{shurFormationLowCenteredIceWedge2025}. 
Resolving whether these features are fundamentally aggradational (as discussed above) or degradational (opposite theory) is particular important for understanding the potential mobilization and transport of radionuclides within \glspl{hcp}, which are, under either theory, the definitive result of permafrost degradation.

As in the theory discussed here, \glspl{lcp} need sufficient cold temperatures to grow, it is obvious to assume that the growth rate of an \gls{lcp} tends to be reduced with increasing winter temperatures. 
However, it also must be considered how changes in insulation by snow and vegetation over top of ice wedges modulate the ground temperatures, lower snow insulation could therefore favor \glspl{lcp} growth even with higher but still sufficient low temperatures.

Several studies (e.g.\citep{fraserRecentPondingUpland2023,steedmanSpatioTemporalVariationHighCentre2017}) have observed accelerated degradation of \glspl{iwp}, primarily indicated by increased melt pond formation in the Western Canadian Arctic.

\section{Mineral Earth Hummocks}
Mineral earth hummocks are a form of patterned ground, typically appearing as scattered, distinct, and low dome-shaped mounds~\citep{kokeljStructureDynamicsEarth2007}. 
These mounds are generally circular to oval, reaching diameters of up to 2 meters and a relief of about 0.5 meters. 
They are separated by depressed areas called interhummocks, which distance the hummocks from each other by up to 2 meters.

As indicated by the name, mineral earth hummocks consist mainly of mineral soil and are mostly vegetated by low-growing lichens and mosses, if vegetated at all. 
Interhummocks usually contain peat and are dominated by other vegetation as mosses, grasses, and shrubs~\citep{dakinHowDryYear2023,kokeljStructureDynamicsEarth2007,quintonSubsurfaceDrainageHummockcovered2000}.

Besides their differences in vegetation, the active layer depth also differs between hummocks and interhummocks, with up to twice as deep active layer depths for hummocks compared to interhummocks~\citep{kokeljStructureDynamicsEarth2007}.

\citet{mackayOriginHummocksWestern1980} redefined the understanding of hummock formation in permafrost, moving away from a cryostatic theory toward an equilibrium-driven model. 
The foundation of this new understanding is the bowl-shaped frost table that sits beneath the hummock. 
For the short discussion of this theory a two-sided refreezing of the active layer is assumed.

During the freeze-back period, ice lenses form at the top and the bottom of the active layer. 
The geometry of the bowl-shaped permafrost table causes the resulting frost heave to be upward and radially inward for the bottom freeze-back and upward and radially outward for the top freeze-back. 
The overall magnitude of this upward force is modulated by the maximum heat loss through conduction, depending on the surface area, thermal conductivity, and temperature gradient.

During the thaw period, the melting of near-surface ice lenses causes the hummock to subside, pushing the upper material downward. 
As summer progresses, the frost table regains its bowl shape due to the differing thermal properties of the insulating organic soil in the interhummock versus the more conductive mineral soil of the hummock. 
Further melting of ice lenses at the base of the active layer contributes to more downward and inward movement. 
For the hummock to remain stable and exist in an equilibrium state, the total upward heave generated by the refreezing permafrost table must equal the total downward subsidence and outward spreading that occurs during the summer thaw.

\citet{kokeljStructureDynamicsEarth2007} enhanced this theory by shifting the focus of hummock dynamics from a seasonal driven process to a process dominated by long-term changes in the underlain permafrost. 
Their study concluded that the long-term growth and the degradation of hummocks are primarily driven by changes in the ice-rich zone at the top of the permafrost. 
The bowl-shaped permafrost table is still a driving feature for hummock growth, but it is accompanied by the gradual development and enrichment of near-surface ground ice, with ice layers in the interhummock and a high amount of ice lenses below the hummock. 

\citet{kokeljStructureDynamicsEarth2007} also discussed the results of hummock degradation. 
Degraded hummocks are usually wider and flatter than well-developed hummocks. 
This is explained by thaw subsidence and confirmed through field observations. 
Furthermore, there was no bowl-shaped permafrost table found below degraded hummocks, showing the importance of this features for hummocks. 

In conclusion, it was discussed that the dominant mechanism for hummock growth or degradation is not the seasonal soil movement caused by the active layer, but the net heave and settlement associated with permafrost aggradation and degradation, which greatly exceeds the seasonal changes.

Mineral earth hummocks and their interhummocks create an interesting system for subsurface flow analysis. 
Hummocks have a low hydraulic conductivity, due to their high amount of clay and mineral soils and are therefore considered to play a minor role in local hydrology and possible radionuclide release. 
However, the cryoturbations affecting the hummocks lead to the generation of ice lenses below the hummock and the confinement of organic-rich material, creating a possible sink for radionuclides while aggrading.

The interhummocks on the other side are highly conductive and can facilitate relatively fast suprapermafrost flow. 
Therefore, they could efficiently transport radionuclides stored in the peat and the peat hummock transition zone.

A better understanding of the flow pathways in hummock regions and the regularity of hummocks are especially useful for risk assessment and modeling of local contamination spills, as the hummocks modulate the suprapermafrost flow matrix and interhummock characteristics are not yet fully understood~\citep{dakinHowDryYear2023,quintonSubsurfaceDrainageHummockcovered2000}.

\section{Radionuclides in Permafrost}
Radionuclides in the environment originate from both natural and anthropogenic sources. 
Natural radionuclides, such as carbon-14 which is produced in the upper atmosphere, are typically the dominant contributors of total radionuclide concentrations. 
Their formation processes lead to relatively stable concentrations over time, making them useful for applications like radionuclide dating.

In contrast, anthropogenic radionuclides enter the environment through human activities. 
\citet{huSourcesAnthropogenicRadionuclides2010} identified seven pathways for anthropogenic radionuclides, including nuclear weapons testing, nuclear power generation, and uranium mining. 
In regions like the Arctic permafrost zones, radionuclide levels tend to be relatively high~\citep{bondPermafrostThawImplications2018}.

Tritium levels, which peaked during the nuclear weapons testing era (1940–1980), have generally declined over time due to exponential decay~\citep{bondPermafrostThawImplications2018,rozanskiTritiumGlobalAtmosphere1991,schmidtOverviewTritiumRecords2020}. 
However, local variations persist, influenced by nearby sources and geographic factors.

Permafrost thaw poses a potential risk by mobilizing previously stored radionuclides in Arctic soils. 
The primary radioisotopes from nuclear tests are tritium and cesium-137. 
Tritium, being part of water molecules, moves conservatively through hydrological systems, like the active layer. 
In contrast, cesium-137, which is a cation, tends to bind to shallow sediments, making it less mobile and more likely to accumulate in specific areas.
