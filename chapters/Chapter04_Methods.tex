\chapter{Methods}
To meet the objectives discussed different methods will be used at study sites at the \gls{tvc}. 
The location of the station and the adjacent study sites are shown in \cref{fig:study_site_overview} and \cref{fig:study_sites}.

\gls{tvc} is located 50 km north of Inuvik, Northwest Territories and was established in 1991~\citep{trailvalleycreekTrailValleyCreek2025}. 
It is underlain by ice-rich continuous permafrost and offers access to multiple hummocks and \glspl{iwp} sites. While the completion of the \gls{ith} introduced increased anthropogenic influence and reduced isolation due to its proximity (1.5 km from the station), the resulting environmental effects, such as the degradation of snow albedo from road dust, have been quantitatively assessed as having a relatively small overall impact~\citep{hammarSnowAccumulationAlbedo2023}. 
Furthermore, this study focuses on the Siksik Creek watershed, which is a sub-catchment of \gls{tvc} covering an area of 0.5 km2 north to east of \gls{tvc} as well as an instrumented \gls{iwp} site equipped with temperature loggers and ground water wells. 
Due to its extensive long-term data, research legacy, and existing infrastructure it is an explanatory site to improve the understanding of radionuclides in ice-rich continuous permafrost.
\begin{figure}[htbp]
    \centering
    \includegraphics[width=0.8\textwidth]{figures/Overview of Study Site.pdf}
    \caption{Location of \gls{tvc}.}\label{fig:study_site_overview}
\end{figure}
\begin{figure}[htbp]
    \centering
    \includegraphics[width=0.8\textwidth]{figures/Study Sites.pdf}
    \caption[Location of study sites around \gls{tvc}.]{Location of the study sites in the Siksik Creek Watershed, and the ice-wedge polygon. 
        The study sites are named according to their relative location in the watersheds. 
        Dots indicate the location of piezometers, lines indicate the transects used for the \gls{gpr} survey and the \gls{alt} measurements. 
        The air images of the \glspl{iwp} are indicating degenerating \glspl{iwp}. Air images are created by drone on June 2, 2023.}\label{fig:study_sites}
\end{figure}
\section{Standard and Legacy TVC Data and Remote Sensing Data}
The transects, excluding the \gls{iwp} site, are orientated on the transects defined by \citet{dakinHowDryYear2023}. 
Enabling comparative analysis between two different years with each other. 

Meteorological data is provided for both years, collected at the \gls{tvc} meteorological station north of \gls{tvc}, adjacent to the Siksik Middle and Siksik Upper transects, which has been collecting data since 1991. 

Furthermore, lidar data was collected in August 2024 with a point density of 11.5 pulses per square meters corresponding to an average horizontal spacing between points of about 30 cm. 
This data was already used to create a 1 m resolution \gls{dtm} and \gls{dsm}. 
Additionally, there are \glspl{dsm} available created by drone imagery from 2023 with a resolution of 5 cm and will be reproduced with imagery data from 2025.

\section{Active Layer Dynamics}
The standard field method to measure the \gls{alt} is to use a steel probe, also known as active layer probing. 
Here, a steel probe is pushed vertically into the ground, up to the point where the probe hits a significant resistance, which is assumed to be the thaw front, which in its maximum extent equals the ALT.\@
The depth is then measured directly by measuring the length of the probe sticking out of the ground. 
The advantages of this technique are that it is a direct measurement with inexpensive equipment that can be quickly deployed in the field. 
However, there are also disadvantages of this technique. First it is highly labor-intensive and a sufficient balance between resolution and transect/area size must be found. 
Secondly, there is a sampling bias and a subjectivity bias as measurements are highly localized, buried rocks, for example, can be misinterpreted as the thaw front. 
Furthermore, repeated probing can disturb the soil and vegetation, compress the ground and change the thermal regime influencing future surveys.

To mitigate the relatively low resolution of active layer probing a \gls{gpr} can be used~\citep{angelopoulosApplicationCCRGPR2013,guoDetectionPermafrostSubgrade2015,harrisUsingGroundPenetratingRadar2025,hinkelDetectionSubsurfacePermafrost2001,koyanPotential3DGPR2025,moormanImagingNearsurfacePermafrost2007,sudakovaUsingGroundPenetrating2021,sudakovaGroundPenetratingRadarStudies2019}.
\subsection{Ground Penetrating Radar}
Different soil materials have different physical properties. 
Geophysical methods like \gls{gpr} are using such different physical properties to acquire knowledge about the subsurface.

\gls{gpr} identifies the physical boundaries of shallow subsurface structures by detecting variations in electromagnetic properties, such as dielectric permittivity, electrical conductivity, and magnetic permeability. 
The following sections explore the theoretical foundations of \gls{gpr}, drawing primarily from \citet{geoscidevelopersGroundPenetratingRadar2017} online repository, which serves as a valuable resource for gaining deeper insight into the topic.
First, the physical properties relevant to \gls{gpr} are discussed, then the basic principles of \gls{gpr} are outlined, and finally the data acquisition and processing methods used in this study are described.    
\subsubsection{Physical Properties}
\paragraph{Dielectric Permittivity} 
The dielectric permittivity (\(\varepsilon\)) of a material quantifies its ability to be polarized by an external electric field. 
It is the sum of the permittivity of free space (\(\varepsilon_0\)) and the material's specific permittivity (\(\varepsilon_s\)), 
therefore the relative permittivity (\(\varepsilon_r\)) is often used to describe a material's permittivity in relation to free space:
\begin{align}
    \varepsilon_r &= \frac{\varepsilon}{\varepsilon_0}\label{al:varepsilon}
\end{align}

The relative permittivity of a material therefore significantly affects the \gls{gpr} signal's travel time and reflection strength. 
The composition of the materials, such as porosity and water content, influences the relative permittivity. 
As there is a big difference between air (\(\varepsilon_r \approx 1\)), fresh water (\(\varepsilon_r \approx 80\)), and ice (\(\varepsilon_r \approx 3\)) the presence of water or ice in the subsurface has a significant impact on \gls{gpr} signal behavior.
This goes hand in hand with the porosity of a material, as higher porosity allows for more water, ice, or air to be present in the material. 

The dielectric permittivity can be assumed to be constant for all materials, as long as the frequency of the electromagnetic wave is sufficiently low (less than \(\SI{1}{\giga\hertz}\)). 
As this study will use \gls{gpr} equipment that does not exceed \(\SI{1}{\giga\hertz}\), the frequency dependence of dielectric permittivity won't be discussed further.

The relative permittivity of common subsurface materials varies significantly, with air having a value of 1, water around 80, wet clay soils ranging from 20 to 40, and permafrost from 4 to 8~\citep{geoscidevelopersGroundPenetratingRadar2017}.

\paragraph{Electrical Conductivity}
Electrical conductivity (\(\sigma\)) describes the ability of a material to conduct electric current. 
Materials with high electrical conductivity, such as metals, allow electromagnetic waves to penetrate only shallowly due to rapid attenuation. 
Electrical conductivity modifies the penetration depth of \gls{gpr} signals, with higher conductivity leading to reduced penetration depth.

\paragraph{Magnetic Permeability}
The magnetic permeability (\(\mu\)) of a material quantifies its ability to support the formation of a magnetic field within itself. 
It is therefore the magnetic equivalent of dielectric permittivity and similarly influences the behavior of electromagnetic waves.
However, most subsurface materials have magnetic permeabilities close to that of free space. 
Magnetic permeability is therefore often neglected in \gls{gpr} studies.

\subsubsection{Basic Principles of \gls{gpr}}
\gls{gpr} is based on the idea that the subsurface is composed by a set of homogeneous regions that are separeted by boundaries with contrasting electromagnetic properties.

\gls{gpr} systems send pulses of high-frequency electromagnetic waves into the ground. 
Such pulses contain a set of electromagnetic waves oscilating near a particular frequency. 
These waves propagate through the subsurface and interact with the electromagnetic properties (\(\varepsilon\), \(\sigma\), and \(\mu\)) of the subsurface.

When electromagnetic waves encounter a boundary between two materials with different electromagnetic properties, the waves are partially reflected (\(\mathbf{R}\)), transmitted (refracted) (\(\mathbf{T}\)), and absorbed (\(\mathbf{A}\)).
These coefficient depend on the electromagnetic properties of the two materials and the angle of incidence of the wave. 
However, they are related to each other by 
\begin{align}
   |\mathbf{R}| + |\mathbf{T}| + |\mathbf{A}| &= 1\label{al:rtarelationship}\qquad .
\end{align}

\paragraph{Wave Velocity}
The speed of light, which is the velocity of electromagnetic waves, is not constant and depends on the wave's frequency and the physical properties of the medium it is traveling through which are discussed above. 
However, for most \gls{gpr} applications, the wave velocity \(v\) can be approximated by 
\begin{align}
    v &= \frac{1}{\sqrt{\mu \varepsilon}} \approx \frac{c}{\sqrt{\varepsilon_r}} \qquad , \label{al:wave_velocity}
\end{align}
where \(c\) is the speed of light in a vacuum (\(\SI{2.998e8}{\meter\per\second}\)) and a non-magnetic material is assumed.

Equation (\ref{al:wave_velocity}) shows that the wave velocity decreases with increasing relative permittivity.
Meaning that increasing water saturation decreases the wave velocity significantly, because of the high relative permittivity of water. 
While ice has a much lower relative permittivity (\(\varepsilon_r \approx 3\)), explaining why \gls{gpr} is well suited to study (ice-rich) permafrost environments.

\paragraph{Absorption --- Attenuation and Skin Depth}
If a electromagnetic waves propagates through a particular medium, its amplitude \(A\) decreases exponentially with the distance \(d\) traveled:
\begin{align}
    A(d) &= A_0 e^{-\alpha d} \qquad ,\label{al:attenuation}
\end{align}
where \(A_0\) is the initial amplitude and \(\alpha\) is the attenuation coefficient of the medium.

The attenuation coefficient can be approximated by:
\begin{align}
    \alpha \approx \frac{\sigma}{2}\sqrt{\frac{\mu}{\varepsilon}} \qquad ,\label{al:attenuation_coefficient}
\end{align}
where \(\sigma\) is the electrical conductivity, \(\mu\) is the magnetic permeability, and \(\varepsilon\) is the dielectric permittivity of the medium.
This approximation is only valid if the electrical conductivity is low compared to the product of the wave frequency and the dielectric permittivity (\(\sigma \ll \omega \varepsilon\)), which is usually the case for \gls{gpr} applications.

The skin depth \(\delta\) is defined as the depth at which the amplitude of the wave falls to \(1/e\) of its original value:
\begin{align}
    \delta &= \frac{1}{\alpha} \qquad ,\label{al:skin_depth}
\end{align}
indicating how deeply the electromagnetic waves can penetrate into the medium before being attenuated by 37\% of its original amplitude.
The skin depth is generally smaller for higher frequencies, however, for the wave regime approximation (\(\sigma \ll \omega \varepsilon\)) the skin depth is frequency independent.

In conclusion, the skin depth increases with decreasing electrical conductivity and increases with increasing dielectric permittivity.

\paragraph{Reflection --- Polarity and Total Reflection}
The reflection coefficient \(\mathbf{R}\) is defined as the ratio of the amplitude of the reflected wave to the incident amplitude and is given by:
\begin{align}
    \mathbf{R} &= \frac{\sqrt{\varepsilon_{r1}} - \sqrt{\varepsilon_{r2}}}{\sqrt{\varepsilon_{r1}} + \sqrt{\varepsilon_{r2}}}\label{al:reflection_coefficient}
\end{align}
Where \(\varepsilon_{r1}\) is the relative permittivity of the medium carrying the incoming wave and \(\varepsilon_{r2}\) is the relative permittivity of the second medium.

From equation (\ref{al:reflection_coefficient}) it can be seen that the reflection coefficient depend on the contrast in relative permittivity between the two media. 
With a higher contrast, the absolute value of the reflection coefficient increases, leading to a stronger reflected wave.
From the reflection coefficient it can also be seen that if the wave is traveling from a medium with lower relative permittivity to a medium with higher relative permittivity (\(\varepsilon_{r2} > \varepsilon_{r1}\)), the reflection coefficient is negative, indicating a phase shift for the reflected wave by \(\pi\) (\(\SI{180}{\degree}\)).

If the wave hits a highly conductive medium, the reflection and transmission of the wave is dominated by the electrical conductivity of the medium. 
In this case, the transmission coefficient approaches zero and a total reflection occurs, regardless of the angle of incidence.

\paragraph{Transmission --- Refraction and Critical Refraction}
The transmission coefficient \(\mathbf{T}\) is defined as the ratio of the amplitude of the transmitted wave to the incident amplitude and is given by:
\begin{align}
    \mathbf{T} &= \frac{2\sqrt{\varepsilon_{r2}}}{\sqrt{\varepsilon_{r1}} + \sqrt{\varepsilon_{r2}}}\label{al:transmission_coefficient}
\end{align}

If a wave is transmitted from one medium to another with an angle of incidence \(\theta_1\), the wave is refracted according to Snell's law:
\begin{align}
    \frac{\sin \theta_1}{v_1} &= \frac{\sin \theta_2}{v_2} \label{al:snells_law} \\
    \Leftrightarrow \sqrt{\varepsilon_1} \sin \theta_1 &= \sqrt{\varepsilon_2} \sin \theta_2 \\
    \Leftrightarrow \frac{v_1}{v_2} &= \frac{\sin \theta_1}{\sin \theta_2} = \sqrt{\frac{\varepsilon_2}{\varepsilon_1}} \qquad ,
\end{align}
where \(v_1\) and \(v_2\) are the wave velocities in the first and second medium, respectively, \(\theta_2\) is the angle of refraction in the second medium, and \(\varepsilon_1\) and \(\varepsilon_2\) are the dielectric permittivities of the first and second medium, respectively.

This also implies that the wave will bend towards the normal when entering a medium with a higher dielectric permittivity (\(\varepsilon_2 > \varepsilon_1\)) and away from the normal when entering a medium with a lower dielectric permittivity (\(\varepsilon_2 < \varepsilon_1\)). 
Meaning that the the wave will either be focused or `spread out' when entering the second medium.

There is a critical angle of incidence \(\theta_c\) where the angle of refraction \(\theta_2\) becomes \(\SI{90}{\degree}\) and the wave travels along the boundary between the two media:
\begin{align}
    \sin \theta_c &= \frac{v_2}{v_1} = \sqrt{\frac{\varepsilon_1}{\varepsilon_2}} \qquad ,\label{al:critical_angle}
\end{align}
which can only occur when the wave is traveling from a medium with higher dielectric permittivity to a medium with lower dielectric permittivity (\(\varepsilon_1 > \varepsilon_2\)).

\paragraph{Scattering}
In contrast to the assumption that the subsurface is composed of homogeneous regions, the reality is that the subsurface has a small scale heterogeneity. 
If the dimensions of these heterogeneities are on the order of 1/4th of the wavelength of the electromagnetic wave, scattering occurs. 
Sources of scattering can be irregular surface shapes of larger buried objects, rocky soils, but also gas bubbles trapped in ice. 

\paragraph{Geometric Spreading}
As the electromagnetic wave propagates through the subsurface, the energy of the wave spreads out over an increasing area. 
This geometric spreading causes a decrease in the amplitude of the wave with increasing distance from the source. 
Meaning that the detection of deeper reflections requires higher initial signal strength, limiting the penetration depth of \gls{gpr} systems.

\subsubsection{Data Acquisition and Processing}
For this study a \SI{500}{\mega\hertz} and \SI{1}{\giga\hertz} \gls{gpr} system from Sensors \& Software Inc.\ will be used. 
The antennas will be used in a common offset configuration, where the transmitter and receiver antennas are separated by a fixed distance. 
The \gls{gpr} system will be mounted in a sled and pulled along the transects shown in \cref{fig:study_sites}, one survey will be run while the surface is still snow covered (late April) and a second survey will be conducted in mid-August, when the \gls{alt} is close to its maximum. 
Every transect will be surveyed with both antennas two times, creating a start to end and end to start survey.
The \gls{gpr} data will be processed using the ReflexW software from Sandmeier. 
And \gls{dsm}, \gls{dtm}, and airborne photographs from drone surveys, as well as the ground samples will be used to interpret the processed data. 

\section{Suprapermafrost Water Observations}
Groundwater wells will be installed in a grid like system on a sloped mineral earth hummock side (see \cref{fig:study_sites}). 
Close to these wells, soil core samples will be taken and classified to improve the \gls{gpr} interpretation. 
Furthermore, the groundwater wells will be equipped with water level loggers to estimate the saturated layer and the groundwater temperature. 

This information will be used to create a piezometric surface of the active layer mapping active layer flow.

Between the groundwater wells, surface temperature, the subsurface temperature at around 5 cm depth and the surface soil moisture will be measured weekly. 
Providing information about soil insulation and soil reaction to weather changes.
\section{Radionuclide composition}
Water samples of groundwater from the groundwater wells and surface water from Siksik and \gls{tvc} will be analyzed for their radionuclide composition, for their general water chemistry, and their stable isotope composition. 
Results will be interpreted in the context of active layer flow system, flow paths in the interhummocks and connection with the downgradient surface water bodies. 
