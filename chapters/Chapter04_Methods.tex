\chapter{Methods}
To meet the objectives discussed different methods will be used at study sites at the Trail Valley Creek Research Station (TVC). 
The location of the station and the adjacent study sites are shown in \cref{fig:study_site_overview} and \cref{fig:study_sites}.

TVC is located 50 km north of Inuvik, Northwest Territories and was established in 1991~\citep{trailvalleycreekTrailValleyCreek2025}. 
It is underlain by ice-rich continuous permafrost and offers access to multiple hummocks and IWP sites While the completion of the Inuvik-Tuktoyaktuk Highway (ITH) introduced increased anthropogenic influence and reduced isolation due to its proximity (1.5 km from the station), the resulting environmental effects, such as the degradation of snow albedo from road dust, have been quantitatively assessed as having a relatively small overall impact~\citep{hammarSnowAccumulationAlbedo2023}. 
Furthermore, this study focuses on the Siksik Creek watershed, which is a sub-catchment of Trail Valley Creek covering an area of 0.5 km2 north to east of TVC as well as an instrumented IWP site equipped with temperature loggers and ground water wells. 
Due to its extensive long-term data, research legacy, and existing infrastructure it is an explanatory site to improve the understanding of radionuclides in ice-rich continuous permafrost.
\begin{figure}[htbp]
    \centering
    \includegraphics[width=0.8\textwidth]{figures/Overview of Study Site.pdf}
    \caption{Location of TVC.}
    \label{fig:study_site_overview}
\end{figure}
\begin{figure}[htbp]
    \centering
    \includegraphics[width=0.8\textwidth]{figures/Study Sites.pdf}
    \caption{Location of the study sites in the Siksik Creek Watershed, and the ice-wedge polygon. 
            The study sites are named according to their relative location in the watersheds. 
            Dots indicate the location of piezometers, lines indicate the transects used for the ground penetrating radar survey and the active layer thickness measurements. 
            The air images of the IWPs are indicating degenerating IWPs. Air images are created by drone on June 2, 2023.}
    \label{fig:study_sites}
\end{figure}
\section{Standard and Legacy TVC Data and Remote Sensing Data}
The transects, excluding the IWP site, are orientated on the transects defined by \citet{dakinHowDryYear2023}. 
Enabling comparative analysis between two different years with each other. 

Meteorological data is provided for both years, collected at the TVC meteorological station north of TVC, adjacent to the Siksik Middle and Siksik Upper transects, which has been collecting data since 1991. 

Furthermore, lidar data was collected in August 2024 with a point density of 11.5 pulses per square meters corresponding to an average horizontal spacing between points of about 30 cm. 
This data was already used to create a 1m resolution digital terrain (DTM) and digital surface model (DSM). 
Additionally, there are DSMs available created by drone imagery from 2023 with a resolution of 5 cm and will be reproduced with imagery data from 2025.

\section{Active Layer Dynamics}
The standard field method to measure the active layer thickness (ALT) is to use a steel probe, also known as active layer probing. 
Here, a steel probe is pushed vertically into the ground, up to the point where the probe hits a significant resistance, which is assumed to be the thaw front, which in its maximum extent equals the ALT. 
The depth is then measured directly by measuring the length of the probe sticking out of the ground. 
The advantages of this technique are that it is a direct measurement with inexpensive equipment that can be quickly deployed in the field. 
However, there are also disadvantages of this technique. First it is highly labor-intensive and a sufficient balance between resolution and transect/area size must be found. 
Secondly, there is a sampling bias and a subjectivity bias as measurements are highly localized, buried rocks, for example, can be misinterpreted as the thaw front. 
Furthermore, repeated probing can disturb the soil and vegetation, compress the ground and change the thermal regime influencing future surveys.

To mitigate the relatively low resolution of active layer probing a ground penetrating radar (GPR) can be used~\citep{angelopoulosApplicationCCRGPR2013,guoDetectionPermafrostSubgrade2015,harrisUsingGroundPenetratingRadar2025,hinkelDetectionSubsurfacePermafrost2001,koyanPotential3DGPR2025,moormanImagingNearsurfacePermafrost2007,sudakovaUsingGroundPenetrating2021,sudakovaGroundPenetratingRadarStudies2019}.
\subsection{Ground Penetrating Radar}
Different soil materials have different physical properties. 
Geophysical methods like GPR are using such different physical properties to acquire knowledge about the subsurface.

GPR identifies the physical boundaries of shallow subsurface structures by detecting variations in electromagnetic properties, such as dielectric permittivity, electrical conductivity, and magnetic permeability. 
The following sections explore the theoretical foundations of GPR, drawing primarily from \citet{geoscidevelopersGroundPenetratingRadar2017} online repository, which serves as a valuable resource for gaining deeper insight into the topic.
\subsubsection{Physical Properties}

\section{Suprapermafrost Water Observations}
Groundwater wells will be installed in a grid like system on a sloped mineral earth hummock side (see \cref{fig:study_sites}). 
Close to these wells, soil core samples will be taken and classified to improve the GPR interpretation. 
Furthermore, the groundwater wells will be equipped with water level loggers to estimate the saturated layer and the groundwater temperature. 

This information will be used to create a piezometric surface of the active layer mapping active layer flow.

Between the groundwater wells, surface temperature, the subsurface temperature at around 5 cm depth and the surface soil moisture will be measured weekly. 
Providing information about soil insulation and soil reaction to weather changes.
\section{Radionuclide composition}
Water samples of groundwater from the groundwater wells and surface water from Siksik and Trail Valley Creek will be analyzed for their radionuclide composition, for their general water chemistry, and their stable isotope composition. 
Results will be interpreted in the context of active layer flow system, flow paths in the interhummocks and connection with the downgradient surface water bodies. 
