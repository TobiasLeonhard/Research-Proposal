\chapter{Objectives}
This research has two objectives which will both be used to build the foundation of a database that can be used to verify modeling outputs, reflecting micro-topographic features like mineral earth hummocks, and \glspl{iwp}. 
\section{Objective 1: Investigating the influence of microtopographic features on active layer development and suprapermafrost flow}
The first objective focuses on characterizing how microtopographic structures -—- specifically mineral earth hummocks and \glspl{iwp} -—- modulate the seasonal dynamics of the active layer in continuous permafrost. 
This study aims to distinguish thaw depth variations and assess their relationship to subsurface flow pathways, by using a combination of \gls{gpr}, active layer probing, and high-resolution surface imagery. 
By integrating temperature and moisture data with spatial mapping, the research will provide insight into how these features influence water movement and thermal insulation, contributing to a more nuanced understanding of permafrost hydrology.
\section{Objective 2: Establishing baseline concentrations and spatial variability of radionuclides in Arctic permafrost environments}
The second objective is to quantify the concentrations of tritium and cesium-137 in water and soil samples collected from hummock sites. 
These radionuclides are assumed to be remnants of mid-20th century atmospheric nuclear testing. 
By analyzing their distribution in relation to active layer flow and microtopographic controls, this study will provide foundational data for future modeling efforts and risk assessments related to radionuclide mobility in Arctic landscapes.
