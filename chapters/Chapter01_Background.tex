\chapter{Background and Motivation}
Diesel generators are the primary power source for more than 200 remote northern and Arctic communities and northern mining facilities. 
These diesel generators present cost and logistical challenges due to fuel delivery by ice roads, airplanes, and boats, and increase Canada's greenhouse gas emissions. 
The Canadian Government has developed a roadmap to consider the implementation of small nuclear modular reactors in those remote communities and mining facilities to 
reduce electricity costs and carbon emissions~\citep{canadiansmallmodularreactorroadmapsteeringcommitteeCallActionCanadian2018}.

The northern and Arctic regions consist of permafrost, which is ground material that is at or below 0°C for at least two consecutive years~\citep{lewkowiczIllustratedPermafrostDictionary2024}. 
Permafrost is known to store and accumulate legacy fallout radionuclides and other pollutants~\citep{bondPermafrostThawImplications2018, olsonMercuryCyclingNorthern2018}, 
a better understanding of the mobility of radionuclides could therefore improve the overall understanding of the mobility of pollutants in permafrost. 
Additionally, Arctic lake sediments also store radionuclides~\citep{marshLakeSedimentationMackenzie1999}. 
However, the Arctic Monitoring and Assessment Programme (AMAP)~\citep{arcticmonitoringandassessmentprogrammeamapAMAPAssessment20152015} reported that there is considerable uncertainty about radionuclide levels in the various environments of the Canadian Arctic.

It is anticipated that the radioecology in the Arctic will change as the climate in the Arctic changes~\citep{arcticmonitoringandassessmentprogrammeamapAMAPAssessment20152015, bondPermafrostThawImplications2018}. 
However, it is unclear how the radioecology in the Arctic will respond to permafrost thaw and changes in snow cover, soil moisture, vegetation, thermokarst lake development, ground temperature, and streamflow~\citep{overlandIntegratedIndexRecent2019}. 
Furthermore, measurements of radionuclide levels in the Arctic are generally relatively rare and do not allow an understanding of the spatial variability in radionuclides in areas of ice-rich continuous permafrost.

Microtopographic features, such as mineral earth hummocks, however, are big constraints for active layer flow~\citep{quintonSubsurfaceDrainageHummockcovered2000}, and are therefore possible features that may influence the uptake and storage of radionuclides. 
Remote sensing can be used to estimate the horizontal extent of such features~\citep{dakinHowDryYear2023}, as long as the vegetation cover does not hinder this approach, however, the vertical extent is also important to estimate possible flow paths and the interconnectedness of the inter-hummock zones.

To estimate the vertical extent better, high resolution data for active layer and organic layer depths are needed. 
In recent years ground penetrating radar was increasingly used to map features in the near earth surface, including active layer depth, organic layer depth, and ice-wedges~\citep{angelopoulosApplicationCCRGPR2013, guoDetectionPermafrostSubgrade2015, harrisUsingGroundPenetratingRadar2025, hinkelDetectionSubsurfacePermafrost2001, koyanPotential3DGPR2025, sudakovaUsingGroundPenetrating2021, sudakovaGroundPenetratingRadarStudies2019}.

This research will improve the international understanding of radionuclides in ice-rich continuous permafrost through the combination of ground penetrating radar, high resolution surface images, active layer flow mapping, and water and soil samples. 
