\chapter{Conclusion}
The transition from diesel generators to \glspl{smr} in remote northern and Arctic communities and mining facilities presents a promising solution to reduce both logistical challenges and greenhouse gas emissions. 
However, the unique environmental conditions of the Arctic, particularly the presence of permafrost and its role in storing radionuclides, necessitate a thorough understanding of the potential impacts on radioecology with changes in the active layer and permafrost. 
This research aims to enhance our knowledge of how changes in the Arctic environment could affect radionuclide release and its subsequent impact on ecosystems. 
By investigating the structure and behavior of the active layer, soil composition, and active layer flow, this study will provide valuable insights for risk assessment and the sustainable implementation of \glspl{smr} in these sensitive regions and provide a better estimate of radionuclide background levels in the Canadian Arctic.
